	
\chapter{Simulation Environment}
%The chapter gives a description of the simulation environment we have created. Presenting how we created it, why we needed to create it and motivation of the choices made. First we give an short overview of the parts that are in the enviroment, and a definition of in/out data. Motivating all our choices made concerning limitations in the enviroment, and also describing positive features of our enviroment.\\
In order to attain the data needed for a comparison of our different algorithms it was nessecary to construct a good testing enviroment. We decided to create this enviroment by the use of two separate parts. One part is called the "Map generator". This part creates a map of the enviroment, tests the feasibility and prints feasible enviroments into an output file. The other part is called the "Network generator". This part reads in an enviroment from a file, creates a graph network to the corresponding map and gives each node in the network its relevant information.\\

\section{Map generator}
The Map Generator (MG) creates random feasible enviroments. A feasible enviroment is described more in detail in section 2, but in short one could say that an enviroment is feasible if it is simply connected and can be divided into a finite set of convex regions. Given the desired size and the density (percentage of obstacles per total area) as inputs, MG creates square shaped feasible enviroment with randomly placed obstacles and saves the map in an external file. For simplicity we have chosen to construct enviroments consisting only of square regions. We suggest that this does not result in a loss of generality since one can construct any feasible enviroment by a sufficiently fine meshing of squares. \\

pseudo-code, Map Generator:\\
input variables:\\
int size; //Specifies the width and height of the square matrix  A.\\
int NumberOfEnv; // Specifies how many feasible enviroments to create.\\
int Obstacle; //  Specifies the number of obstacles in percents, e.g. number of obstacles per total area of A.\\
\\
while (created feasible enviroments is less than the variable NumberOfEnv )\\
	Create a matrix A with all elements set to one; \\
	in a random maner, turn elements in A to zero (corresponding to obstacles) until the desired amount of obstacles have been placed.\\
	Test if A is connected\\
	if A is connected write the matrix A to the file OK.out\\
	if A is not connected write the matrix to a file NotOK.out\\


\section{Node network}
The node network generator has a matrix that represents the environment as input. The matrix can either be generated by the simulation environment generator, se previous section, or written by hand.\\
\par{} A node network will be generated from the input matrix. Each element in the input matrix is represented by a node, which is given the attributes Vision, move and state. Vision is an array of all the nodes that can be seen from the current node, move is an array with the neighbour nodes to the current node.\\
\\
Pseudo-code:\\
* Read environment matrix from file\\
* Create a node-matrix B, and give each node a default state\\
* Calculate and set move and vision attributes for each node\\